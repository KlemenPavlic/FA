% Začetek preambule
\documentclass[a4paper, 12pt]{article}
\usepackage[slovene]{babel}
\usepackage[utf8]{inputenc}
\usepackage[T1]{fontenc}
\usepackage{lmodern}
\usepackage{amsfonts}

% Moji ukazi, okolja,...
% oznake za števila
\newcommand{\N}{\mathbb{N}}
\newcommand{\Z}{\mathbb{Z}}
\newcommand{\Q}{\mathbb{Q}}
\newcommand{\R}{\mathbb{R}}
\newcommand{\C}{\mathbb{C}}
\newcommand{\F}{\mathbb{F}}
% opombe
\newenvironment{opomba}{\begin{flushleft} \textbf{Opomba}:}{\hfill \end{flushleft}}
% definicije
\newcounter{definitionCounter}
\addtocounter{definitionCounter}{1}
\newenvironment{definicija}{\begin{flushleft} \textit{\textbf{Definicija \arabic{definitionCounter}}}:}{\hfill \end{flushleft}\stepcounter{definitionCounter}}
% pojmi
\newcommand{\pojem}[1]{\textsc{#1}}
% dokazi
\newenvironment {dokaz}{\begin{flushleft} \textit{\textbf{Dokaz}}:}{\hfill $\square$\end{flushleft}}
% izreki
\newcounter{theoremCounter}
\addtocounter{theoremCounter}{1}
\newcounter{theoremCorollaryCounter}
\addtocounter{theoremCorollaryCounter}{0}
\newenvironment {izrek}{\begin{flushleft} \textsf{\textbf{IZREK \arabic{theoremCounter}}}:}{\hfill \end{flushleft}\stepcounter{theoremCounter}\stepcounter{theoremCorollaryCounter}\setcounter{corollaryCounter}{1}}
% leme
\newcounter{lemmaCounter}
\addtocounter{lemmaCounter}{1}
\newenvironment{lema}{\begin{flushleft} \textbf{Lema \arabic{lemmaCounter}}:}{\hfill \end{flushleft}\stepcounter{lemmaCounter}}
% posledice
\newcounter{corollaryCounter}
\addtocounter{corollaryCounter}{1}
\newenvironment  {posledica}{\begin{flushleft} \textsf{\textbf{Posledica \arabic{theoremCorollaryCounter}.\arabic{corollaryCounter}}}:}{\hfill \end{flushleft}\stepcounter{corollaryCounter}}
% dodatni ukazi
\usepackage{hyperref} % mora biti zadnji

% začetek dokumenta
\begin{document}
\begin{flushright}
Klemen Pavlič

27122002
\end{flushright}
\begin{center}
Funkcionalna analiza
\end{center}
\begin{center}
PRVA DOMAČA NALOGA
\end{center}

\begin{flushleft}
1. naloga
\end{flushleft}
Na Banachovem prostoru $\mathcal{C}[0,1]$ je definirana preslikava $A$ s predpisom
$$
(Af)(x) = (x-x^2) \int_{0}^x f(t) dt.
$$
\begin{enumerate}
\item[(a)] Dokaži, da je $A$ dobro definiran omejen linearen operator in izračunaj njegovo normo.
\item[(b)] Ali je operator $A$ injektiven?
\item[(c)] Ali je operator $A$ obrnljiv?
\end{enumerate}
\emph{Rešitev}

\begin{flushleft}
2. naloga
\end{flushleft}
Naj bo $(X,\mathcal{A}, \mu )$ merljiv prostor, $1 \le p < \infty$ in $A\subseteq X$ merljiva podmnožica v $X$. Definirajmo
$$
L^p(A) = \{ f\in L^p(X): f = 0 \textrm{ skoraj povsod na } A^C\}.
$$
\begin{enumerate}
\item[(a)] Dokaži, da je $L^p(A)$ zaprt podprostor v $L^p(X)$.
\item[(b)] Kateremu prostoru je izometrično izomorfen $L^p(X) / L^p(A)$?
\end{enumerate}
\emph{Rešitev}

\begin{flushleft}
3. naloga
\end{flushleft}
Naj za vektorja $x$ in $y$ normiranega prostora velja $|| x+y|| = ||x|| + ||y||$. Dokaži, da za poljubni nenegativni števili $\alpha$ in $\beta$ velja
$$
||\alpha x + \beta y|| = \alpha ||x|| + \beta ||y||.
$$
\emph{Rešitev}
\newline
Predpostavimo lahko, da je $\alpha \ge \beta \ge 0$ (sicer zamenjamo vlogi). Potem velja 
$$
\alpha ||x|| + \beta ||y|| = ||\alpha x|| + ||\beta y|| \ge ||\alpha x + \beta y ||= ||\alpha x + \alpha y - \alpha y + \beta y||=
$$
$$
||\alpha (x+y) - (\alpha - \beta)y|| \ge \Big| ||\alpha(x+y) || - ||(\alpha - \beta)y|| \Big| = \Big| \alpha||x+y|| - (\alpha - \beta) ||y|| \Big|=
$$
$$
\Big| \alpha (||x|| + ||y||) - \alpha ||y|| + \beta ||y||\Big| = \Big| \alpha ||x|| + \beta ||y||\Big| = \alpha ||x|| + \beta ||y||.
$$
Sledi $||\alpha x + \beta y || = \alpha ||x|| + \beta ||y||$.

\begin{flushleft}
4. naloga
\end{flushleft}
Na normiranem prostoru $c_{00}$ (vseh vektorjev $x\in l^{\infty}$, ki imajo končno mnogo neničelnih komponent) je podano zaporedje $\{f_n\}_{n\in \N}$ linearnih funkcionalov s predpisom
$$
f_n(x_1,x_2,\dots) = n x_n.
$$
\begin{enumerate}
\item[(a)] Dokaži, da je za vsak $x\in c_{00}$ zaporedje vektorjev $\{ f_n(x) \}_{n\in \N}$ omejeno.
\item[(b)] Ali je zaporedje $\{ || f_n || \}_{n\in \N}$ omejeno?
\item[(c)] Ali je to v protislovju z izrekom o enakomerni omejenosti?
\end{enumerate}
\emph{Rešitev}
\begin{enumerate}
\item[(a)] Naj bo $x=(x_1,x_2,\dots) \in c_{00}$. Naj bo $\mathcal{I}$ množica tistih indeksov, pri katerih ima vektor $x$ neničelne komponente, tj. $\mathcal{I} = \{ i \in \N; x_i \neq 0\}$. Potem je po predpostavki $|\mathcal{I} | < \infty$. Označimo z $M = \max \mathcal{I}$. Za člene zaporedja $\{f_n(x)\}_{n\in \N}$ velja $f_n(x) = 0$, če je $n\in \mathcal{I}^C$. V tem zaporedju je torej le končno mnogo neničenih členov (to so ravno tisti pri katerih je indeks $n\in \mathcal{I}$). Za te pa velja $|f_n(x)| = |nx_n| =n |x_n| \le M ||x||$. Sledi, da je zaporedje $\{f_n(x)\}_{n\in \N}$ omejeno za poljben $x\in c_{00}$.

\item[(b)] Vemo, da je $||f_n|| = \sup\{|f_n(x)|, ||x||=1\}$. Naj bo vektor $e_i$ tak, da ima na $i$-tem mestu enico, povsod drugje pa ničle. Potem je jasno $e_i\in c_{00}$ in $||e_i||=1$. Ker je $|f_n(e_n)| = n\cdot 1$, je zagotovo $||f_n|| \ge n$. Ker to velja za vsak $n\in \N$, zaporedje $\{||f_n||\}_{n\in \N}$ ni omejeno.

\item[(c)] To ni v protislovju z izrekom o enakomerni omejenosti, saj ta velja le za Banachove prostore. Trdim namreč, da prostor $c_{00}$ ni Banachov. To sledi, ker ni zaprt. Res, naj bo $x_{n} = (1,\frac{1}{2},\dots,\frac{1}{n},0,\dots)$. Potem je zaporedje $\{x_n\}_{n\in\N}$ zaporedje vektorjev iz $c_{00}$, ki očitno konvergira proti vektorju $x=(x_i)_{i\in \N}, x_i = \frac{1}{i}$, ki pa ni v $c_{00}$, saj nima le končno neničelnih komponent. 
\end{enumerate}


\begin{flushleft}
5. naloga
\end{flushleft}
Naj bo $X$ Banachov prostor in $\mathcal{F}, \mathcal{G}$ števni podmnožici v $\mathcal{B}(X)$. Naj za vsak $x\in X$ obstajata taka operatorja $T\in \mathcal{F}$ in $S\in \mathcal{G}$, da je $Tx = Sx$. Dokaži, da je $\mathcal{F} \cap \mathcal{G} \neq \emptyset$.
\newline
\emph{Rešitev}


\begin{flushleft}
6. naloga
\end{flushleft}
Naj bo $A:X\rightarrow Y$ surjektiven omejen linearen operator med Banachovima prostoroma $X$ in $Y$.
\begin{enumerate}	
\item[(a)] Dokaži, da obstaja taka konstanta $M> 0$, da za vsak $y\in Y$ obstaja vsaj en vektor $x\in X$, za katerega velja $Ax = y$ in $||x|| \le M ||y||$.
\item[(b)] Naj bo $\{y_n\}_{n\in \N}$ zaporedje v $Y$, ki konvergira proti 0. Dokaži, da obstaja tako zaporedje $\{x_n\}_{n\in \N}$ v $X$, ki konvergira proti 0 in za vsak $n\in \N$ velja $Ax_n = y_n$.
\end{enumerate}
\emph{Rešitev}

\end{document}