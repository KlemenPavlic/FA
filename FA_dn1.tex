% Začetek preambule
\documentclass[a4paper, 12pt]{article}
\usepackage[slovene]{babel}
\usepackage[utf8]{inputenc}
\usepackage[T1]{fontenc}
\usepackage{lmodern}
\usepackage{amsfonts}

% Moji ukazi, okolja,...
% oznake za števila
\newcommand{\N}{\mathbb{N}}
\newcommand{\Z}{\mathbb{Z}}
\newcommand{\Q}{\mathbb{Q}}
\newcommand{\R}{\mathbb{R}}
\newcommand{\C}{\mathbb{C}}
\newcommand{\F}{\mathbb{F}}
% opombe
\newenvironment{opomba}{\begin{flushleft} \textbf{Opomba}:}{\hfill \end{flushleft}}
% definicije
\newcounter{definitionCounter}
\addtocounter{definitionCounter}{1}
\newenvironment{definicija}{\begin{flushleft} \textit{\textbf{Definicija \arabic{definitionCounter}}}:}{\hfill \end{flushleft}\stepcounter{definitionCounter}}
% pojmi
\newcommand{\pojem}[1]{\textsc{#1}}
% dokazi
\newenvironment {dokaz}{\begin{flushleft} \textit{\textbf{Dokaz}}:}{\hfill $\square$\end{flushleft}}
% izreki
\newcounter{theoremCounter}
\addtocounter{theoremCounter}{1}
\newcounter{theoremCorollaryCounter}
\addtocounter{theoremCorollaryCounter}{0}
\newenvironment {izrek}{\begin{flushleft} \textsf{\textbf{IZREK \arabic{theoremCounter}}}:}{\hfill \end{flushleft}\stepcounter{theoremCounter}\stepcounter{theoremCorollaryCounter}\setcounter{corollaryCounter}{1}}
% leme
\newcounter{lemmaCounter}
\addtocounter{lemmaCounter}{1}
\newenvironment{lema}{\begin{flushleft} \textbf{Lema \arabic{lemmaCounter}}:}{\hfill \end{flushleft}\stepcounter{lemmaCounter}}
% posledice
\newcounter{corollaryCounter}
\addtocounter{corollaryCounter}{1}
\newenvironment  {posledica}{\begin{flushleft} \textsf{\textbf{Posledica \arabic{theoremCorollaryCounter}.\arabic{corollaryCounter}}}:}{\hfill \end{flushleft}\stepcounter{corollaryCounter}}
% dodatni ukazi
\usepackage{hyperref} % mora biti zadnji

% začetek dokumenta
\begin{document}
\begin{flushright}
Klemen Pavlič

27122002
\end{flushright}
\begin{center}
Funkcionalna analiza
\end{center}
\begin{center}
PRVA DOMAČA NALOGA
\end{center}

\begin{flushleft}
1. naloga
\end{flushleft}
Na Banachovem prostoru $\mathcal{C}[0,1]$ je definirana preslikava $A$ s predpisom
$$
(Af)(x) = (x-x^2) \int_{0}^x f(t) dt.
$$
\begin{enumerate}
\item[(a)] Dokaži, da je $A$ dobro definiran omejen linearen operator in izračunaj njegovo normo.
\item[(b)] Ali je operator $A$ injektiven?
\item[(c)] Ali je operator $A$ obrnljiv?
\end{enumerate}
\emph{Rešitev}
\begin{enumerate}
\item[(a)]Ker je $f$ zvezna funkcija, integral $\int_0^x f(t)dt$ obstaja za vsak $x\in[0,1]$, velja še več, to je celo zvezna funkcija v spremenljivki $x$ (zgornja meja). Ker je tudi $x-x^2$ zvezna funkcija, je $Af$ zvezna, saj je produkt dveh zveznih funkcij. Sledi, da operator $A$ res slika v prostor $\mathcal{C}[0,1]$. Ker pa je še integral linearen, lahko zaključimo, da je $A$ dobro definiran linearen operator. Velja še 
$$
|Af(x)| = |(x-x^2)\int_0^x f(t)dt| = |x-x^2| |\int_0^x f(t)dt| \le
$$
$$
(x-x^2)||f||_{\infty} \int_0^x 1 dt= (x^2 - x^3) ||f||_{\infty} \le \frac{4}{27} ||f||_{\infty}.
$$
Pripomnimo še, da zadnji neenačaj velja, ker ima funkcija $x\mapsto x^2 - x^3$ v točki $x=\frac{2}{3}$ maksimum. Tam doseže vrednost $\frac{4}{27}$. To se z lahkoto poračuna, zato bom ta delček izpustil.
 Zgornja ocena nam pove še, da je operator $A$ omejen in da je $||A|| \le \frac{4}{27}$. Velja $A1(x) = x^2 - x^3$ in enak premislek kot zgoraj nam pove, da je $||A1|| = \max\{x^2 - x^3, x\in[0,1]\}=\frac{4}{27}$. To implicira še $||A||\ge \frac{4}{27}$. Zaključimo, da je $||A||=\frac{4}{27}$.

\item[(b)] Denimo, da obstajata dve funckiji $f$ in $g$, da je $Af = Ag$. To pomeni, da za vsak $x\in [0,1]$ velja $(x-x^2) \int_0^x f(t)dt = (x-x^2) \int_0^xg(t)dt$. Od tod sledi, da mora biti že $\int_0^x f(t)dt = \int_0^x g(t)dt$. Ker pa sta to odvedljivi funkciji, morati biti tudi njuna odvoda enaka. To implicira, da je $g(x) = f(x)$ za vsak $x\in[0,1]$, torej $g=f$. Sledi, da je $A$ injektiven operator.

\item[(c)] Trdim, da $A$ ni obrnljiv. Če bi namreč bil, bi to impliciralo, da je tudi surjektiven, kar pa očitno ni, saj za vsako funkcijo v njegovi sliki velja, da ima v točki $x=0$ vrednost 0.
\end{enumerate}

\begin{flushleft}
2. naloga
\end{flushleft}
Naj bo $(X,\mathcal{A}, \mu )$ merljiv prostor, $1 \le p < \infty$ in $A\subseteq X$ merljiva podmnožica v $X$. Definirajmo
$$
L^p(A) = \{ f\in L^p(X): f = 0 \textrm{ skoraj povsod na } A^C\}.
$$
\begin{enumerate}
\item[(a)] Dokaži, da je $L^p(A)$ zaprt podprostor v $L^p(X)$.
\item[(b)] Kateremu prostoru je izometrično izomorfen $L^p(X) / L^p(A)$?
\end{enumerate}
\emph{Rešitev}
\begin{enumerate}
\item[(a)] Pokažimo, da je $L^p(A)$ zaprt podprostor. Jasno je, da je podprostor. Pokažimo, da je zaprt. Naj bo $\{f_n\}_{n\in \N} \subseteq L^p(A)$ konvergentno zaporedje. Ker je $L^p(X)$ poln, obstaja $f\in L^p(X)$, da zaporedje $\{f_n\}_{n\in\N}$ konvergira proti $f$ v $p$-normi. To pomeni, da za vsak $\epsilon > 0$ obstaja $n_0$, da za vsak $n\ge n_0$ velja $||f_n - f||_p < \epsilon^{1/p}$ oziroma $||f_n - f||_p^p < \epsilon$. To pomeni, da je 
$$
\epsilon > \int_X |f_n - f|^p d\mu = \int_A |f_n - f|^p d\mu + \int_{A^C} |f_n-f|^pd\mu, 
$$
od koder sledi, da je 
$$
\epsilon > \int_{A^C} |f_n-f|^pd\mu = \int_{A^C} |f|^pd\mu, 
$$
kjer zadnji enačaj velja, ker so $f_n \in L^p(A)$. Ker je bil $\epsilon>0$ poljubno majhen to pomeni, da je $|f|$ enaka 0 skoraj povsod na $A^C$. Posledično je tudi $f$ enaka 0 skoraj povsod na $A^C$. Sledi, da je $f\in L^p(A)$ in zato je to zaprt podprostor. 

\item[(b)] Trdim, da je $L^p(X) / L^p(A)$ izometrično izomorfen $L^p(A^C)$, kjer je $L^p(A^C)=\{f\in L^p(X): f=0\textrm{ skoraj povsod na } A\}$. Pripomnil bi, da lahko podobno kot zgoraj vidimo, da je tudi $L^p(A^C)$ zaprt podprostor $L^p(X)$ in zato Banachov.
Definirajmo preslikavo
$$
\varphi:L^p(X) /L^p(A) \rightarrow L^p(A^C), \quad \varphi(f+L^p(A)) = f\chi_{A^C}.
$$ 
Trdim, da je ta preslikava izometrični izomorfizem. Ker je $A$ merljiva množica, je tudi $A^C$ merljiva. Sledi, da je karakteristična preslikava $\chi_{A^C}$ merljiva in ker je produkt merljivih preslikav spet merljiva, je tudi $f\chi_{A^C}$ merljiva. Iz predpisa je tudi razvidno, da $f\chi_{A^C}$ zadošča pogoju za pripadnost prostoru $L^p(A^C)$. Če za odseka velja $f_1 + L^p(A) = f_2 + L^p(A)$, je to ekvivalentno temu, da je $f_1 - f_2 \in L^p(A)$, kar pomeni, da je $f_1 - f_2 = 0$ skoraj povsod na $A^C$. Sledi, da je preslikava $\varphi$ res dobro definirana in da res slika v $L^p(A^C)$. Jasno je, da je $\varphi$ linearna. Če je $\varphi(f+L^p(A)) = f\chi_{A^C} = 0$, je $f=0$ skoraj povsod na $A^C$, torej je $f\in L^p(A)$. To implicira, da je $\varphi$ injektivna. Ker je $L^p(A^C)$ podprostor v $L^p(X)$, je $\varphi$ surjektivna. Res, za vsak $f\in L^p(A^C) \subseteq L^p(X)$ namreč velja $\varphi (f +L^p(A)) = f\chi_{A^C} = f$.
Preveriti moramo le še, da je $\varphi$ res izometrija. Zadošča preveriti, da je 
$$
||f\chi_{A^C}||^p = ||\varphi(f+L^p(A))||^p = ||f+L^p(A)||^p = \inf\{||f-g||^p, g\in L^p(A)\}.
$$
Naj bo zdaj $g\in L^p(A)$ poljuben. Potem velja
$$
||f-g||^p = \int_X |f-g|^p d\mu = \int_A|f-g|^p d\mu + \int_{A^C} |f|^p d\mu.
$$
V zgornji vsoti želimo minimizirati še prvi sumand, zato izberemo optimalni element $g=f\chi_A\in L^p(A)$. Zgornja vsota je  naprej enaka
$$
\int_A|f-f|^p d\mu + \int_{A^C} |f|^pd\mu = \int_{A^C}|f|^pd\mu = ||f\chi_{A^C}||^p.
$$
Ker je bil izbrani $g$ optimalni v smislu, da minimizira normo $||f-h||$ po $h\in L^p(A)$, je $\varphi$ res izometrija.
\end{enumerate}

\begin{flushleft}
3. naloga
\end{flushleft}
Naj za vektorja $x$ in $y$ normiranega prostora velja $|| x+y|| = ||x|| + ||y||$. Dokaži, da za poljubni nenegativni števili $\alpha$ in $\beta$ velja
$$
||\alpha x + \beta y|| = \alpha ||x|| + \beta ||y||.
$$
\emph{Rešitev}
\newline
Predpostavimo lahko, da je $\alpha \ge \beta \ge 0$ (sicer zamenjamo vlogi). Potem velja 
$$
\alpha ||x|| + \beta ||y|| = ||\alpha x|| + ||\beta y|| \ge ||\alpha x + \beta y ||= ||\alpha x + \alpha y - \alpha y + \beta y||=
$$
$$
||\alpha (x+y) - (\alpha - \beta)y|| \ge \Big| ||\alpha(x+y) || - ||(\alpha - \beta)y|| \Big| = \Big| \alpha||x+y|| - (\alpha - \beta) ||y|| \Big|=
$$
$$
\Big| \alpha (||x|| + ||y||) - \alpha ||y|| + \beta ||y||\Big| = \Big| \alpha ||x|| + \beta ||y||\Big| = \alpha ||x|| + \beta ||y||.
$$
Sledi $||\alpha x + \beta y || = \alpha ||x|| + \beta ||y||$.

\begin{flushleft}
4. naloga
\end{flushleft}
Na normiranem prostoru $c_{00}$ (vseh vektorjev $x\in l^{\infty}$, ki imajo končno mnogo neničelnih komponent) je podano zaporedje $\{f_n\}_{n\in \N}$ linearnih funkcionalov s predpisom
$$
f_n(x_1,x_2,\dots) = n x_n.
$$
\begin{enumerate}
\item[(a)] Dokaži, da je za vsak $x\in c_{00}$ zaporedje vektorjev $\{ f_n(x) \}_{n\in \N}$ omejeno.
\item[(b)] Ali je zaporedje $\{ || f_n || \}_{n\in \N}$ omejeno?
\item[(c)] Ali je to v protislovju z izrekom o enakomerni omejenosti?
\end{enumerate}
\emph{Rešitev}
\begin{enumerate}
\item[(a)] Naj bo $x=(x_1,x_2,\dots) \in c_{00}$. Naj bo $\mathcal{I}$ množica tistih indeksov, pri katerih ima vektor $x$ neničelne komponente, tj. $\mathcal{I} = \{ i \in \N; x_i \neq 0\}$. Potem je po predpostavki $|\mathcal{I} | < \infty$. Označimo z $M = \max \mathcal{I}$. Za člene zaporedja $\{f_n(x)\}_{n\in \N}$ velja $f_n(x) = 0$, če je $n\in \mathcal{I}^C$. V tem zaporedju je torej le končno mnogo neničenih členov (to so ravno tisti pri katerih je indeks $n\in \mathcal{I}$). Za te pa velja $|f_n(x)| = |nx_n| =n |x_n| \le M ||x||$. Sledi, da je zaporedje $\{f_n(x)\}_{n\in \N}$ omejeno za poljben $x\in c_{00}$.

\item[(b)] Vemo, da je $||f_n|| = \sup\{|f_n(x)|, ||x||=1\}$. Naj bo vektor $e_i$ tak, da ima na $i$-tem mestu enico, povsod drugje pa ničle. Potem je jasno $e_i\in c_{00}$ in $||e_i||=1$. Ker je $|f_n(e_n)| = n\cdot 1$, je zagotovo $||f_n|| \ge n$. Ker to velja za vsak $n\in \N$, zaporedje $\{||f_n||\}_{n\in \N}$ ni omejeno.

\item[(c)] To ni v protislovju z izrekom o enakomerni omejenosti, saj ta velja le za Banachove prostore. Trdim namreč, da prostor $c_{00}$ ni Banachov. To sledi, ker ni zaprt. Res, naj bo $x_{n} = (1,\frac{1}{2},\dots,\frac{1}{n},0,\dots)$. Potem je zaporedje $\{x_n\}_{n\in\N}$ zaporedje vektorjev iz $c_{00}$, ki očitno konvergira proti vektorju $x=(x_i)_{i\in \N}, x_i = \frac{1}{i}$, ki pa ni v $c_{00}$, saj nima le končno neničelnih komponent. 
\end{enumerate}


\begin{flushleft}
5. naloga
\end{flushleft}
Naj bo $X$ Banachov prostor in $\mathcal{F}, \mathcal{G}$ števni podmnožici v $\mathcal{B}(X)$. Naj za vsak $x\in X$ obstajata taka operatorja $T\in \mathcal{F}$ in $S\in \mathcal{G}$, da je $Tx = Sx$. Dokaži, da je $\mathcal{F} \cap \mathcal{G} \neq \emptyset$.
\newline
\emph{Rešitev}
\newline
Označimo s $T_i, i=1,2,\dots,$ elemente množice $\mathcal{F}$ in s $S_j, j=1,2,\dots$, elemente množice $\mathcal{G}$. Definiramo množice $A_{ij} = \{x\in X: T_i x = S_jx\} = \ker(T_i - S_j)$. Množice $A_{ij}$ so zaprte, saj gre ravno za jedro omejenega linearnega operatorja $T_i-S_j$. Očitno jih je tudi števno. Ker za vsak $x\in X$ obstajata operatorja $T_i$ in $S_j$, da je $T_i x = S_j x$, velja še $\cup_{i,j=1}^{\infty} A_{ij} = X$. Po posledici Baireovega izreka obstajata indeksa $i_0, j_0$, da ima množica $A_{i_0 j_0}$ neprazno notranjost, torej vsebuje neko odprto kroglo $K(x_0,r),r>0$. Na njej se operatorja $T_{i_0}$ in $S_{j_0}$ ujemata. Naj bo sedaj $x\in K(0,r)$. Tedaj je $T_{i_0} x = T_{i_0}(x_0 + x - x_0)= T_{i_0} (x_0 + x) - T_{i_0}(x_0)= S_{j_0}(x_0+x)-S_{j_0}(x_0) = S_{j_0}(x)$, torej se operatorja $T_{i_0}$ in $S_{j_0}$ ujemata na odprti krogli $K(0,r)$. Naj bo zdaj še $x\in X$ poljuben. Tedaj je $T_{i_0}x = \frac{2||x||}{r}T_{i_0}(\frac{r}{2||x||}x) =\frac{2||x||}{r}S_{j_0}(\frac{r}{2||x||}x) = S_{j_0}x$, kar pomeni, da se $T_{i_0}$ in $S_{j_0}$ ujemata na celem $X$. To pomeni, da sta enaka, torej je $T_{i_0} = S_{j_0}\in \mathcal{F}\cap \mathcal{G}\neq \emptyset$.


\begin{flushleft}
6. naloga
\end{flushleft}
Naj bo $A:X\rightarrow Y$ surjektiven omejen linearen operator med Banachovima prostoroma $X$ in $Y$.
\begin{enumerate}	
\item[(a)] Dokaži, da obstaja taka konstanta $M> 0$, da za vsak $y\in Y$ obstaja vsaj en vektor $x\in X$, za katerega velja $Ax = y$ in $||x|| \le M ||y||$.
\item[(b)] Naj bo $\{y_n\}_{n\in \N}$ zaporedje v $Y$, ki konvergira proti 0. Dokaži, da obstaja tako zaporedje $\{x_n\}_{n\in \N}$ v $X$, ki konvergira proti 0 in za vsak $n\in \N$ velja $Ax_n = y_n$.
\end{enumerate}
\emph{Rešitev}
\begin{enumerate}
\item[(a)]
\item[(b)]
\end{enumerate}

\end{document}